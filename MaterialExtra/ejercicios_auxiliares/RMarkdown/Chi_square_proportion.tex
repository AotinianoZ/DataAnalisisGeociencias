% Options for packages loaded elsewhere
\PassOptionsToPackage{unicode}{hyperref}
\PassOptionsToPackage{hyphens}{url}
%
\documentclass[
]{article}
\usepackage{lmodern}
\usepackage{amssymb,amsmath}
\usepackage{ifxetex,ifluatex}
\ifnum 0\ifxetex 1\fi\ifluatex 1\fi=0 % if pdftex
  \usepackage[T1]{fontenc}
  \usepackage[utf8]{inputenc}
  \usepackage{textcomp} % provide euro and other symbols
\else % if luatex or xetex
  \usepackage{unicode-math}
  \defaultfontfeatures{Scale=MatchLowercase}
  \defaultfontfeatures[\rmfamily]{Ligatures=TeX,Scale=1}
\fi
% Use upquote if available, for straight quotes in verbatim environments
\IfFileExists{upquote.sty}{\usepackage{upquote}}{}
\IfFileExists{microtype.sty}{% use microtype if available
  \usepackage[]{microtype}
  \UseMicrotypeSet[protrusion]{basicmath} % disable protrusion for tt fonts
}{}
\makeatletter
\@ifundefined{KOMAClassName}{% if non-KOMA class
  \IfFileExists{parskip.sty}{%
    \usepackage{parskip}
  }{% else
    \setlength{\parindent}{0pt}
    \setlength{\parskip}{6pt plus 2pt minus 1pt}}
}{% if KOMA class
  \KOMAoptions{parskip=half}}
\makeatother
\usepackage{xcolor}
\IfFileExists{xurl.sty}{\usepackage{xurl}}{} % add URL line breaks if available
\IfFileExists{bookmark.sty}{\usepackage{bookmark}}{\usepackage{hyperref}}
\hypersetup{
  pdftitle={Chi Cuadrado Iguales Proporciones},
  pdfauthor={Estudiantes de UNI},
  hidelinks,
  pdfcreator={LaTeX via pandoc}}
\urlstyle{same} % disable monospaced font for URLs
\usepackage[margin=1in]{geometry}
\usepackage{color}
\usepackage{fancyvrb}
\newcommand{\VerbBar}{|}
\newcommand{\VERB}{\Verb[commandchars=\\\{\}]}
\DefineVerbatimEnvironment{Highlighting}{Verbatim}{commandchars=\\\{\}}
% Add ',fontsize=\small' for more characters per line
\usepackage{framed}
\definecolor{shadecolor}{RGB}{248,248,248}
\newenvironment{Shaded}{\begin{snugshade}}{\end{snugshade}}
\newcommand{\AlertTok}[1]{\textcolor[rgb]{0.94,0.16,0.16}{#1}}
\newcommand{\AnnotationTok}[1]{\textcolor[rgb]{0.56,0.35,0.01}{\textbf{\textit{#1}}}}
\newcommand{\AttributeTok}[1]{\textcolor[rgb]{0.77,0.63,0.00}{#1}}
\newcommand{\BaseNTok}[1]{\textcolor[rgb]{0.00,0.00,0.81}{#1}}
\newcommand{\BuiltInTok}[1]{#1}
\newcommand{\CharTok}[1]{\textcolor[rgb]{0.31,0.60,0.02}{#1}}
\newcommand{\CommentTok}[1]{\textcolor[rgb]{0.56,0.35,0.01}{\textit{#1}}}
\newcommand{\CommentVarTok}[1]{\textcolor[rgb]{0.56,0.35,0.01}{\textbf{\textit{#1}}}}
\newcommand{\ConstantTok}[1]{\textcolor[rgb]{0.00,0.00,0.00}{#1}}
\newcommand{\ControlFlowTok}[1]{\textcolor[rgb]{0.13,0.29,0.53}{\textbf{#1}}}
\newcommand{\DataTypeTok}[1]{\textcolor[rgb]{0.13,0.29,0.53}{#1}}
\newcommand{\DecValTok}[1]{\textcolor[rgb]{0.00,0.00,0.81}{#1}}
\newcommand{\DocumentationTok}[1]{\textcolor[rgb]{0.56,0.35,0.01}{\textbf{\textit{#1}}}}
\newcommand{\ErrorTok}[1]{\textcolor[rgb]{0.64,0.00,0.00}{\textbf{#1}}}
\newcommand{\ExtensionTok}[1]{#1}
\newcommand{\FloatTok}[1]{\textcolor[rgb]{0.00,0.00,0.81}{#1}}
\newcommand{\FunctionTok}[1]{\textcolor[rgb]{0.00,0.00,0.00}{#1}}
\newcommand{\ImportTok}[1]{#1}
\newcommand{\InformationTok}[1]{\textcolor[rgb]{0.56,0.35,0.01}{\textbf{\textit{#1}}}}
\newcommand{\KeywordTok}[1]{\textcolor[rgb]{0.13,0.29,0.53}{\textbf{#1}}}
\newcommand{\NormalTok}[1]{#1}
\newcommand{\OperatorTok}[1]{\textcolor[rgb]{0.81,0.36,0.00}{\textbf{#1}}}
\newcommand{\OtherTok}[1]{\textcolor[rgb]{0.56,0.35,0.01}{#1}}
\newcommand{\PreprocessorTok}[1]{\textcolor[rgb]{0.56,0.35,0.01}{\textit{#1}}}
\newcommand{\RegionMarkerTok}[1]{#1}
\newcommand{\SpecialCharTok}[1]{\textcolor[rgb]{0.00,0.00,0.00}{#1}}
\newcommand{\SpecialStringTok}[1]{\textcolor[rgb]{0.31,0.60,0.02}{#1}}
\newcommand{\StringTok}[1]{\textcolor[rgb]{0.31,0.60,0.02}{#1}}
\newcommand{\VariableTok}[1]{\textcolor[rgb]{0.00,0.00,0.00}{#1}}
\newcommand{\VerbatimStringTok}[1]{\textcolor[rgb]{0.31,0.60,0.02}{#1}}
\newcommand{\WarningTok}[1]{\textcolor[rgb]{0.56,0.35,0.01}{\textbf{\textit{#1}}}}
\usepackage{graphicx,grffile}
\makeatletter
\def\maxwidth{\ifdim\Gin@nat@width>\linewidth\linewidth\else\Gin@nat@width\fi}
\def\maxheight{\ifdim\Gin@nat@height>\textheight\textheight\else\Gin@nat@height\fi}
\makeatother
% Scale images if necessary, so that they will not overflow the page
% margins by default, and it is still possible to overwrite the defaults
% using explicit options in \includegraphics[width, height, ...]{}
\setkeys{Gin}{width=\maxwidth,height=\maxheight,keepaspectratio}
% Set default figure placement to htbp
\makeatletter
\def\fps@figure{htbp}
\makeatother
\setlength{\emergencystretch}{3em} % prevent overfull lines
\providecommand{\tightlist}{%
  \setlength{\itemsep}{0pt}\setlength{\parskip}{0pt}}
\setcounter{secnumdepth}{-\maxdimen} % remove section numbering

\title{Chi Cuadrado Iguales Proporciones}
\author{Estudiantes de UNI}
\date{10/1/2021}

\begin{document}
\maketitle

\hypertarget{chi-cuadrado-iguales-proporciones}{%
\section{Chi Cuadrado Iguales
Proporciones}\label{chi-cuadrado-iguales-proporciones}}

Es un test para poder determinar si \textbf{las proporciones
poblacionales} son iguales.

Debemos tener en cuenta la funcion \texttt{chisq.test()} para evaluar.
Ver formula en la web de \textbf{Williams}\footnote{\url{https://sites.williams.edu/bklingen/files/2012/02/R-code-for-inference-about-several-proportions.pdf}}.

\emph{Ejemplo7}:

\begin{Shaded}
\begin{Highlighting}[]
\NormalTok{table =}\StringTok{ }\KeywordTok{matrix}\NormalTok{(}\KeywordTok{c}\NormalTok{(}\DecValTok{115}\NormalTok{,}\DecValTok{53}\NormalTok{,}\DecValTok{40}\NormalTok{,}\DecValTok{98}\NormalTok{,}\DecValTok{35}\NormalTok{,}\DecValTok{22}\NormalTok{,}\DecValTok{35}\NormalTok{,}\DecValTok{40}\NormalTok{,}\DecValTok{8}\NormalTok{,}\DecValTok{5}\NormalTok{,}\DecValTok{4}\NormalTok{,}\DecValTok{5}\NormalTok{),}\DataTypeTok{ncol=}\DecValTok{4}\NormalTok{,}\DataTypeTok{byrow=}\OtherTok{TRUE}\NormalTok{)}
\NormalTok{table}
\end{Highlighting}
\end{Shaded}

\begin{verbatim}
##      [,1] [,2] [,3] [,4]
## [1,]  115   53   40   98
## [2,]   35   22   35   40
## [3,]    8    5    4    5
\end{verbatim}

\newpage

\begin{Shaded}
\begin{Highlighting}[]
\KeywordTok{colnames}\NormalTok{(table)=}\KeywordTok{c}\NormalTok{(}\StringTok{"AAHH1"}\NormalTok{,}\StringTok{"AAHH2"}\NormalTok{,}\StringTok{"AAHH3"}\NormalTok{,}\StringTok{"AAHH4"}\NormalTok{)}
\KeywordTok{rownames}\NormalTok{(table)=}\KeywordTok{c}\NormalTok{(}\StringTok{"De acuerdo"}\NormalTok{,}\StringTok{"En desacuerdo"}\NormalTok{,}\StringTok{"No opina"}\NormalTok{)}
\NormalTok{table}
\end{Highlighting}
\end{Shaded}

\begin{verbatim}
##               AAHH1 AAHH2 AAHH3 AAHH4
## De acuerdo      115    53    40    98
## En desacuerdo    35    22    35    40
## No opina          8     5     4     5
\end{verbatim}

\begin{Shaded}
\begin{Highlighting}[]
\KeywordTok{chisq.test}\NormalTok{(table)}
\end{Highlighting}
\end{Shaded}

\begin{verbatim}
## Warning in chisq.test(table): Chi-squared approximation may be incorrect
\end{verbatim}

\begin{verbatim}
## 
##  Pearson's Chi-squared test
## 
## data:  table
## X-squared = 14.042, df = 6, p-value = 0.02917
\end{verbatim}

\begin{Shaded}
\begin{Highlighting}[]
\KeywordTok{chisq.test}\NormalTok{(table)}\OperatorTok{$}\NormalTok{expected}
\end{Highlighting}
\end{Shaded}

\begin{verbatim}
## Warning in chisq.test(table): Chi-squared approximation may be incorrect
\end{verbatim}

\begin{verbatim}
##                    AAHH1     AAHH2     AAHH3    AAHH4
## De acuerdo    105.104348 53.217391 52.552174 95.12609
## En desacuerdo  45.339130 22.956522 22.669565 41.03478
## No opina        7.556522  3.826087  3.778261  6.83913
\end{verbatim}

Esta advertencia se produce porque muchos o algunos de los valores
esperados son muy pequeños y las aproximaciones del \textbf{p-valor}
pueden no ser \emph{correctas},para limpiar esto debemos usar el
argumento de la funcion \textbf{chisq.test()} el cual es
\texttt{simulate.p.value\ =\ TRUE}.

\begin{Shaded}
\begin{Highlighting}[]
\KeywordTok{chisq.test}\NormalTok{(table,}\DataTypeTok{simulate.p.value=}\OtherTok{TRUE}\NormalTok{, }\DataTypeTok{B=}\DecValTok{3000}\NormalTok{)}
\end{Highlighting}
\end{Shaded}

\begin{verbatim}
## 
##  Pearson's Chi-squared test with simulated p-value (based on 3000
##  replicates)
## 
## data:  table
## X-squared = 14.042, df = NA, p-value = 0.02199
\end{verbatim}

Otro connatacion que existe es que se asume que la data es normalmente
distribuida, si es pequeña la aproximación esto puede ser pobre.

\textbf{Nota}:

*El valor de estadístico de prueba se aproxima a una distribución χ2c,
si n≥30 y todas frecuencias esperadas Ei son mayores que 5 (en ocasiones
deberemos agrupar varias categorías a fin de que se cumpla este
requisito).

\begin{verbatim}
*Tener en cuenta que se puede realizar el test de Fisher.
\end{verbatim}

Pero para la solucion nuestra juntaremos dos filas:

\begin{Shaded}
\begin{Highlighting}[]
\NormalTok{table2 =}\KeywordTok{matrix}\NormalTok{(}\KeywordTok{c}\NormalTok{(}\DecValTok{115}\NormalTok{,}\DecValTok{50}\NormalTok{,}\DecValTok{40}\NormalTok{,}\DecValTok{98}\NormalTok{,}\DecValTok{43}\NormalTok{,}\DecValTok{27}\NormalTok{,}\DecValTok{39}\NormalTok{,}\DecValTok{45}\NormalTok{),}\DataTypeTok{ncol=}\DecValTok{4}\NormalTok{,}\DataTypeTok{byrow=}\OtherTok{TRUE}\NormalTok{)}
\NormalTok{table2}
\end{Highlighting}
\end{Shaded}

\begin{verbatim}
##      [,1] [,2] [,3] [,4]
## [1,]  115   50   40   98
## [2,]   43   27   39   45
\end{verbatim}

\begin{Shaded}
\begin{Highlighting}[]
\KeywordTok{colnames}\NormalTok{(table2)=}\KeywordTok{c}\NormalTok{(}\StringTok{"AAHH1"}\NormalTok{,}\StringTok{"AAHH2"}\NormalTok{,}\StringTok{"AAHH3"}\NormalTok{,}\StringTok{"AAHH4"}\NormalTok{)}
\KeywordTok{rownames}\NormalTok{(table2)=}\KeywordTok{c}\NormalTok{(}\StringTok{"De acuerdo"}\NormalTok{,}\StringTok{"En desacuerdo/No opina"}\NormalTok{)}
\NormalTok{table2}
\end{Highlighting}
\end{Shaded}

\begin{verbatim}
##                        AAHH1 AAHH2 AAHH3 AAHH4
## De acuerdo               115    50    40    98
## En desacuerdo/No opina    43    27    39    45
\end{verbatim}

\begin{Shaded}
\begin{Highlighting}[]
\KeywordTok{chisq.test}\NormalTok{(table2)}
\end{Highlighting}
\end{Shaded}

\begin{verbatim}
## 
##  Pearson's Chi-squared test
## 
## data:  table2
## X-squared = 12.036, df = 3, p-value = 0.007261
\end{verbatim}

\end{document}
